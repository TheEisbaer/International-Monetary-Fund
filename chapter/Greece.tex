\section{Greece and the International Monetary Fund}
\label{sec:greece}
The most well-known \gls{IMF} lending arrangement is that with Greece. Greece applied for a total of three loans between 2010 and 2017, two of which got approved. In total Greece applied for \gls{SDR} 51.5 billion, which is about 4600\% of their quota – out of these applied loans Greece only used \gls{SDR} 27.74 billion and has an outstanding debt of \gls{SDR} 4.55419 billion. 
Their first application for a loan of \gls{SDR} 26.4 billion was in 2010. The policies imposed on Greece were to transform Greece into an export-led growth mode and to suppress domestic demand with wage and benefits cuts. Greece implemented these policies quickly, but the \gls{IMF} underlined the need for more reforms, such as cuts to health spending and wages, privatizations and abolishment of trade barriers. Greece cancelled the SBA arrangement and borrowed (and repaid) a total of \gls{SDR} 17.54 billion out of the initially agreed \gls{SDR} 26.4 billion.\cite{WikipediaGreeceand2020}
In 2012 Greece applied for their second loan, an Extended Fund Facility, of \gls{SDR} 23.7853 billion. The \gls{IMF} imposed a policy program consisting of the following policies: improvement of competitiveness and growth, reduction of minimum wage and public spending, fight of tax evasion. After the Greek elections, the new government implements the programs and “made impressive progress under the new coalition government”.\cite{InternationalMonetaryFundIMFSurvey2013} After the elections in January of 2015 the \gls{IMF} expressed, that the new Greek government is not willing to implement agreed upon policies. Greece then misses a repayment of \gls{SDR} 1.2 billion as the first developed country to do so.\cite{CasertElenaBecatorosandRafGreecefails2015} The agreement was cancelled, the Greek government drew \gls{SDR} 10.2 billion out of the initially agreed \gls{SDR} 23.8 billion. Later, the \gls{IMF} published a study claiming Greek debt will peek at 200\% of GDP in the following years.\cite{InternationalMonetaryFund.EuropeanDept2015}
Greece’s latest application for an \gls{IMF} Stand-By Agreement was in 2017. The loan was supposed to help Greece recover from the former crises but was not approved by the \gls{IMF}, claiming “Greece’s debt remains unsustainable”\cite{InternationalMonetaryFundIMFExecutive2017}
